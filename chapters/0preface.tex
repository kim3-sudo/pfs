\chapter*{Preface}
This book is meant to be used in a first course in programming and computer science using the Python programming language. It takes the Python-as-a-tool approach by using practical examples and introducing real-world libraries that increase Python's capabilities to scientists. It assumes no prior programming experience and no mathematics experience beyond high school algebra.
\subsection*{Flexibility in Topic Ordering}
This book was written to allow instructors latitude in reordering the material. To illustrate the versatility in material ordering, here is an alternative material flow.
\subsubsection*{Reordering: Variables First}
In order to understand how variables work and how they are stored, a student needs to understand the different datatypes. The basic material covered in Chapter 3 covers the primitive datatypes in Python, whereas variables are covered in Chapter 4.
\subsection*{Accessibility to Students}
It is not enough for a book to simply present the right topics in the right order. It is not even enough for it to be clear and correct when read by an instructor or some other experienced programmer, like a tutor. It's important that the material is presented in a way that is accessible to beginning students. However, subsequent versions of this textbook may improve readability. The authors and publishers encourage students and instructors alike to submit suggestions and recommendations for improving the readability of the content for future editions.
\subsection*{Edition Compatibility}
All efforts will be made to make different editions compatible. Future editions will be written with two or three sections. For example, a version of the first edition may be written as 1.3.1. However, the book and subsequent editions are written in a way that a student could read any version of the first edition. That is, versions 1.0.0, 1.1.0, and 1.1.1 should all be intercompatible by material. Page numbers may differ slightly. We would encourage instructors to reference material based on chapter number, rather than page number, especially because page numbers do not line up between the instructor's edition and the student edition.\par
\subsection*{Printing}
When printing this book, we \textbf{\textit{highly recommend}} printing in color and at 100\% scaling to ensure readability of code. Code snips in this book have been color-coded using color coding, a common feature found in many IDEs, including Visual Studio Code and Atom. At the very least, the colors will allow students to read and distinguish different elements of code.\par
\subsection*{Standards and Platforms}
This edition is fully compatible with base Python 3 (tested up to Python 3.6). Additional packages include Pandas v.1, BeautifulSoup4 v4.9, and TensorFlow v.2. This book is intended to be operating system-agnostic. Where we provide operating-system-specific guidance, we will provide it for Windows 8.1 or later, macOS 10.12 or later, \textit{and} Linux (based on Debian 8 or later and RHEL 7 or later).
\subsection*{Support Material}
There is support material available to all users of this book and additional material available only to qualified instructors. All material is provided free of charge and under the MIT License.
\subsubsection*{Materials Available to All Users}
\begin{itemize}
    \item {Source code from the book}
    \item {Student-led collaborative classroom activities (typically in Process Oriented Guided Inquiry Learning, or POGIL form)}
\end{itemize}
\subsubsection*{Materials Available to Instructors Only}
\begin{itemize}
    \item {Instructor's edition of the text, which includes instructor-specific notes and exercise answers}
    \item {POGIL answer documents}
\end{itemize}
\subsection*{Integrated Development Environment (IDE) Recommendations}
This text is written in such a way that instructors can use any integrated development environment that supports Python 3. However, the authors suggest the following IDEs, pending availability.
\begin{itemize}
    \item {\textbf{repl.it\textregistered} is an online and collaborative IDE that runs entirely in the user's web browser.}
    \item {\textbf{Spyder} is part of the Anaconda package for scientific Python development. It runs on an individual computer's hardware. For students who are more familiar with RStudio, it is possible to change the default layout to mimic RStudio with the editor in the top left, console in the bottom left, environment in top right, and help in the bottom right. You can make this change by going to View \textrangle{} Window layouts \textrangle{} Rstudio layout.}
    \item {\textbf{Microsoft\textregistered Visual Studio} is a much more advanced IDE, and being so feature rich, it may be \textit{overpowered} for teaching students how to program. We would recommend using Spyder instead.}
    \item {\textbf{The R Project for Statistical Computing and RStudio} are primarily focused with statistics in R, but R and RStudio have a built-in and very comprehensive Python interpreter called Reticulate (a play on words with reticulated pythons).}
\end{itemize}
\subsection*{Contributing}
I believe that the greatest strength in an open-source textbook is the very fact that anyone can edit and improve it for the greater good. We encourage students and instructors alike to submit their suggested changes and recommendations for future editions of this book. This book will be published in an open repository on GitHub, and we invite changes in the following ways.
\begin{itemize}
    \item {Open an issue and describe what you think should be changed}
    \item {Edit the content yourself, then make a pull/merge request}
    \item {Contact the authors directly with what you think should be changed}
\end{itemize}
\subsection*{License}
The material in this book is licensed under the MIT license. A copy of the license is provided below.\par

\begin{verbatim}
Copyright © 2022 Sejin Kim

Permission is hereby granted, free of charge, to any person
obtaining a copy of this software and associated documentation
files (the "Software"), to deal in the Software without
restriction, including without limitation the rights to use,
copy, modify, merge, publish, distribute, sublicense, and/or
sell copies of the Software, and to permit persons to whom
the Software is furnished to do so, subject to the following
conditions:

The above copyright notice and this permission notice shall
be included in all copies or substantial portions of the
Software.

THE SOFTWARE IS PROVIDED "AS IS", WITHOUT WARRANTY OF ANY KIND,
EXPRESS OR IMPLIED, INCLUDING BUT NOT LIMITED TO THE WARRANTIES
OF MERCHANTABILITY, FITNESS FOR A PARTICULAR PURPOSE AND
NONINFRINGEMENT. IN NO EVENT SHALL THE AUTHORS OR COPYRIGHT
HOLDERS BE LIABLE FOR ANY CLAIM, DAMAGES OR OTHER LIABILITY,
WHETHER IN AN ACTION OF CONTRACT, TORT OR OTHERWISE, ARISING
FROM, OUT OF OR IN CONNECTION WITH THE SOFTWARE OR THE USE OR
OTHER DEALINGS IN THE SOFTWARE.
\end{verbatim}
\subsection*{Acknowledgements}
Many people have assisted me by providing their suggestions, discussions, and other help in preparing this textbook. Much of the work for the zeroth edition of this book was written while I was working with the wonderful people at the Kenyon College Department of Mathematics and Statistics and in the Integrated Program in Humane Studies. I'd also like to extend a special thanks to the individuals who contributed by proofreading and contributing to this text: Dr. James Skon (Professor of Mathematics, Kenyon College), Dr. Nuh Aydin (Professor of Mathematics, Kenyon College), Dr. Bradley Hartlaub (Professor of Mathematics, Kenyon College), Dr. Jon Chun (Professor in the Integrated Program in Humane Studies, Kenyon College), Dr. Katherine Elkins (Professor in the Integrated Program in Humane Studies, Kenyon College), Ashleigh Zarley (LBIS, Kenyon College), Carter Schoenfeld (LBIS, Kenyon College), Josh Katz (Kenyon College), Dev Akre (Kenyon College), and George Novotny (Kenyon College).